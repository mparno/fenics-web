% This template is losely based on the abstract template from Conference
% on Geometry: Theory and Applications (CGTA 2015), and uses code
% listings style ``anslistings'' from the Archive of Numerical Software.
\documentclass[11pt,a4paper]{article}
%\usepackage[square]{natbib}
\usepackage{amsmath}
\usepackage{amsfonts}
\usepackage{amssymb}
\usepackage{graphicx}
\usepackage{algorithm}
\usepackage{algorithmicx}
\usepackage{algpseudocode}
\usepackage{anslistings} % <-- should be provided with this template

\pagestyle{empty}
\usepackage[left=25mm, right=25mm, top=15mm, bottom=20mm, noheadfoot]{geometry}
% please don't change geometry settings!

%\usepackage[T1]{fontenc}   %% get hyphenation and accented letters right
%\usepackage{mathptmx}      %% use fitting times fonts also in formulas

%%%

\begin{document}
\thispagestyle{empty}

% Insert title here:
\title{Title of abstract for the FEniCS'16 workshop}

% Insert authors and afiliations here, speaker in bold: 
\author{Ola Nordmann, University of Oslo, ola.nordmann@mail.edu, \\
\textbf{Kari Nordmann}, University of Oslo, kari.nordmann@mail.edu}

\date{} % please leave date empty
\maketitle\thispagestyle{empty}

% Insert text here:

This document is a simple one-page abstract for the book of abstracts
of FEniCS'16. Accepted abstracts will be collected and published
freely online.
The abstract should be in English. You may wish to include figures,
such as Figure~\ref{fig5}, and a few references, such as \cite{pu} or
\cite{model}, although they should also fit in the 2 page limit.

\begin{equation}
  \int_\Omega \nabla u \cdot \nabla v\, dx = \int_\Omega f v\, dx
\end{equation}

Lorem ipsum dolor sit amet, consectetur adipiscing elit, sed do
eiusmod tempor incididunt ut labore et dolore magna aliqua. Ut enim ad
minim veniam, quis nostrud exercitation ullamco laboris nisi ut
aliquip ex ea commodo consequat. Duis aute irure dolor in
reprehenderit in voluptate velit esse cillum dolore eu fugiat nulla
pariatur. Excepteur sint occaecat cupidatat non proident, sunt in
culpa qui officia deserunt mollit anim id est laborum.

Feel free to include some short illustrative python code snippets:
\begin{python}
a = inner(grad(u), grad(v))*dx
L = dot(f, v)*dx
\end{python}

Or maybe some C++ code if relevant:
\begin{c++}
std::shared_ptr<Mesh> mesh = ...;
\end{c++}

\begin{figure}[h]
	%uncomment next line to include a graphic file
	%\centerline{\includegraphics[width=6cm, angle=-90]{fig5.eps}}
	%and comment out next line
\centerline{\framebox[6cm]{\rule{0cm}{3.5cm} figure example}}
\caption{Sample figure}
\label{fig5}
\end{figure}

\begin{thebibliography}{00}
\addcontentsline{toc}{chapter}{References}
\bibitem{pu} A.A. Milne, \textit{Winnie-the-Pooh}, Methuen, 1926.
\bibitem{model} S.E. Author, H. Buthor, M. Cuthor: Very important topic, 
\textit{The Really Good Journal}, 6 (1969), 501-510.
\end{thebibliography}

\end{document}
